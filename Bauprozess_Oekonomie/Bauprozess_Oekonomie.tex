% Created 2018-05-07 Mon 10:39
\documentclass[11pt]{article}
\usepackage[utf8]{inputenc}
\usepackage[T1]{fontenc}
\usepackage{fixltx2e}
\usepackage{graphicx}
\usepackage{longtable}
\usepackage{float}
\usepackage{wrapfig}
\usepackage{rotating}
\usepackage[normalem]{ulem}
\usepackage{amsmath}
\usepackage{textcomp}
\usepackage{marvosym}
\usepackage{wasysym}
\usepackage{amssymb}
\tolerance=1000
\usepackage{natbib}
\usepackage[linktocpage,pdfstartview=FitH,colorlinks,linkcolor=black,
anchorcolor=black,citecolor=black,filecolor=black,menucolor=black,urlcolor=black]{hyperref}
\usepackage{ngerman}
\author{Christian Sangvik}
\date{\textit{<2018-02-19 Mon>}}
\title{Bauprozess Ökonomie}
\hypersetup{
  pdfkeywords={},
  pdfsubject={},
  pdfcreator={Emacs 25.3.1 (Org mode 8.2.10)}}
\begin{document}

\maketitle
\tableofcontents


\section{Administratives}
\label{sec-1}

\subsection{Semesterplanung}
\label{sec-1-1}

\begin{center}
\begin{tabular}{lrl}
19. Feb & 01 & Editorial\\
26. Feb & 02 & Die Ökonomie der Stadt\\
05. Mär & 03 & Renditen\\
12. Mär & 04 & Chancen \& Risiken\\
19. Mär & -- & Seminarwoche\\
\textbf{26. Mär} & 05 & \textbf{benotete Übung I}\\
02. Apr & -- & Osterferien\\
09. Apr & 06 & Kosten\\
16. Apr & -- & Sechseläuten\\
23. Apr & 07 & Projektentwicklung\\
30. Apr & 08 & Die Genossenschaft\\
\textbf{07. Mai} & 09 & \textbf{benotete Übung II}\\
14. Mai & 10 & Wrap-Up\\
\end{tabular}
\end{center}

\begin{itemize}
\item zwei benotete schriftliche Übungen während des Semesters
\item Vorlesung ist Grundvoraussetzung für Wahlfacharbeit
\item Kontrollierte Anwesenheitspflicht
\item \href{http://www.bauprozess.arch.ethz.ch/education/MSc/BauprozessOekonomie.html}{Vorlesungsfolien online verfügbar} (werden jeden Montag aufgeschalten)
\end{itemize}

\subsection{Wahlfacharbeit}
\label{sec-1-2}

In diesem Vertiefungsfach ist eine Wahlfacharbeit möglich. Diese wird
prinzipiell als Einzelarbeit geführt, und es ist mit einem Aufwand von
ca. 150 Semesterstunden gerechnet. Die Abgabe der Arbeit erfolgt in
schriftlicher Form.

Genauere Informationen folgen.

\subsection{Literaturempfehlungen}
\label{sec-1-3}

\begin{itemize}
\item Drei Bücher über den Bauprozess (S. Menz)
\item Immobilienökonomie und Bewertung von Liegenschaften (Kaspar Fierz)
\item Die Immobilienbewertung (Francesco Canonica)
\item The Ascent of Money (Niall Ferguson)
\item Capital in the Twenty-First Century (Thomas Pikkety)
\end{itemize}

\subsection{Immobilien Oekonomie App}
\label{sec-1-4}

\href{https://ioe-app.ethz.ch}{IOE App} mit ETH Login verfügbar.

\subsection{Benotete Übungen}
\label{sec-1-5}

\subsubsection{Hilfsmittel}
\label{sec-1-5-1}



\section{Vorlesungen}
\label{sec-2}

\subsection{Editorial}
\label{sec-2-1}

\subsubsection{Notizen}
\label{sec-2-1-1}

5\% Regel. Whg muss mit 5\% Zinsen noch finanzierbar sein.

Das Bauen hinkt immer ca. zwei Jahre hinter der Konjunktur her, da das Bauen
an sich langsam ist. Somit kann man nicht direkt auf den Markt reagieren.

Der Grossteil der Bauaufgaben wird auf Rendite und aus ökonomischen Motiven
erteilt.

\begin{enumerate}
\item Wie entstehen Werte?
\label{sec-2-1-1-1}

Wie bestimmt sich der Grundstückspreis? Was bedeutet dies für uns
Architekten?
\end{enumerate}

\subsubsection{Die Bauökonomie}
\label{sec-2-1-2}

Die Bauökonomie ist in der Architektur bei jedem Projekt von Anfang an im
Fokus. Vo der Aquise bis hin zur Bewirtschaftung ist sie wichtig. Daher ist
es wichtig, sich von Anfang an Gedanken dazu zu machen.

In der SIA Norm 102 kann man aber für die Ökonomie nur den Punkt "Schätzen
des Finanzbedarfes" abrechnen.

In der Praxis brauchen wir aber eine grosse ökonomische Kompetenz.

Variabel ist leider in der Ökonomie nur die Baukosten. Deshalb verkommt der
Architekt häufig zum geometrischen Dienstleister.

\subsubsection{BKP}
\label{sec-2-1-3}

BKP simuliert den Bauprozess.

Der schwierigste Punkt der Schätzung ist das Land.

Neben der Kostenseite gibt es aber auch die Ertragsseite.

\subsection{Die Ökonomie der Stadt, \href{Vorlesungsfolien/02_Die_Oekonomie_der_Stadt_V.pdf}{Folien}}
\label{sec-2-2}

\subsubsection{Einstieg}
\label{sec-2-2-1}

Stadt - Abgeleitetes Wort von Stad, Standort, Stelle.

In der Bewerbung von sehr teuren Bauten beispielsweise in New York, wird dem
Gebäude keine Rechnung mehr getragen. Es wird nur noch mit der Lage
geworben. Baurechtlich sind die gezeigten Bilder aber sehr fraglich, da seit
1916 die ``Wedding Cakes'' in den \emph{zoning laws} vorgeschrieben sind.

In New York kann man aber \emph{air rights} von Nachbarn dazukaufen.

Ähnliches ist in Zürich mit den \emph{Ausnützungstransfers} möglich. So kann man
höher und mit mehr Volumen bauen, als eigentlich vorgesehen.

Die ökonomischen und rechtlichen Prinzipien von New York und Zürich sind
durchaus vergleichbar, obwohl natürlich die Gebäude um einiges niedriger und
kleiner sind.

Ein Ertragssprung findet um das 5. - 7. Geschoss statt, da die
Umgebungsbebauung niedriger ist, und man so einen viel weiteren Ausblick
hat.

Pro Geschoss wird der Wohnraum durchschnittlich um 3.17\% teurer. Dadurch
resultieren circa 25\% mehr Einnahmen.

\subsubsection{Die Stadt als ökonomisches Phänomen}
\label{sec-2-2-2}

Warum die Menschen bereit dazu sind so viel zu bezahlen, um in der Höhe zu
wohnen liegt in der Ökonomischen gegebenheit.

In der Urproduktion waren die Menschen an die Fläche gebunden, um ihr
Überleben zu sichern und genug zu produzieren. Durch technische Entwicklung
und Spezialisierung mit Handel war nicht mehr jeder genötigt den Platz zu
beanspruchen.

Die die Spezialisierung machte auch Schutzmassnahmen notwendig. Schutz
bedeutet Mauer, Turm und Befestigung. Somit waren die Spezialisten, die
Technische und andere Güter herstellten die Begründer der ersten Städte.

Je nach Art der Dienstleistung muss man näher des Zentrums sein. Die
Bereitschaft zu zahlen nimmt aber generell ab, je weiter weg man vom Zentrum
kommt.

Beweggründe hierfür sind diverse Faktoren:

\begin{itemize}
\item Time is money. Je höher der Stundenlohn, desto wichtiger ist die Zentralität.
\end{itemize}

\subsubsection{Fokus Zürich}
\label{sec-2-2-3}

Der Status einer Stadt ist nicht an ihre Grösse gebunden, sondern an ihre
Wichtigkeit in diversen Bereichen. Zürich ist also unproportional wichtig zu
ihrer Grösse.

Zürich scheint relativ effizient auch in den Pendelströmen und Bewegungen.

\begin{enumerate}
\item Auktion: Objekt 1, Gironde
\label{sec-2-2-3-1}

1'096'000.- CHF

\item Auktion: Objekt 2, Seebach
\label{sec-2-2-3-2}

1'050'000.- CHF

\item Erklärungsansätze
\label{sec-2-2-3-3}

Die Kaufkraft in Frankreich ist nicht dieselbe wie in der Schweiz. Man
müsste ungefähr 50\% des Brutto-Median-Einkommens für das Schloss
aufwenden. Das Medianeinkommen in der Gironde ist nur ca. 25'000
CHF/Jahr. In Zürich ist es um 125'000 CHF/Jahr.

Man sollte maximal 30\% des Bruttolohns für Wohnkosten aufwenden.

Das Einkommen und die Kaufkraft machen einen enormen Hebel aus auf die
Kosten der Ligenschaften.

Im Durchschnitt zahlt ein Zürcher aber nur ca 17\% seines
Bruttojahreseinkommens für das Wohnen.

Die Preissteigerung von 3.7\% pro Etage ist aber nur auf die Landkosten
zurückzuführen. Die Erstellungskosten weichen bei weitem nicht so stark
ab.

Das \textbf{Verhältnis} der Anlagekosten zwischen der Erstellungskosten und dem
Landwert heisst Lageklasse. Dieses Verhältnis ist in ähnlichen Lagen immer
dasselbe, obwohl die absoluten Werte stark voneinander abweichen kann.
\end{enumerate}

\subsection{Renditen}
\label{sec-2-3}

\subsubsection{Einstieg}
\label{sec-2-3-1}

Jede fünfte Hypothek wird aufgenommen um dsa Objekt später zu vermieten.

Preise für Eigentumswohnungen sind so hoch, dass man sie eigentlich nicht
gewinnbringend vermieten kann. Die Wertsteigerung des Landes muss höher
sein, als die Abschreibung des Objektes. Der Landwert und die
Immobilienkosten war von der Wirtschaftskriese unberührt. Land und
Immobilien gelten als \emph{cash-cows}.

\subsubsection{Bieterverfahren}
\label{sec-2-3-2}

Je nach Exklusivität wird ein Objekt an einen geschlossenen Kreis von
Investoren ausgeschrieben, sonst generell öffentlich. Architekten gehen
häufig für Investoren an Besichtigungen.

In der zweiten Phase werden die paar höchstbietenden wieder Eingeladen und
eine Finanzierung geprüft.

Vertrauenshaftung [googlen] ist der Architekt haftbar. Man sollte sich
dagegen versichern. Vor allem die rechtliche Situation ist in einem
Vorprojekt und einer Machbarkeitsstudie abzuklären. Sie muss lückenlos
stimmen.

Sanierungsstau prüfen. Sanierungsstau bedeutet, dass der Vorbesitzer nichts
mehr saniert hat mit der Absicht das Objekt sowieso zu verkaufen. Für
umbauten das Baurecht prüfen (Nasszellen, Heizung, Lift, etc.)

Altlasten prüfen.

Bäume sind häufig geschützt. Fällgenehmigung häufig erforderlich.

Risiko auch im Referenzzinssatz. Die Miete muss korrigiert werden. Es kann
sein, dass die Bewohner zu hohe Zinsen zahlen, welche sie korrigieren
müssen.

Erträge eines Jahres = Nettomieten

Glatte Renditezahlen, wie hier mit 5\% kommen daher, dass der Verkäufer oder
dessen Bank eine gute Rendite vorgiebt, um Investoren anzulocken. Sie ist
häufig nicht die Reale rendite. Der Richtverkaufspreis wird dann anhand der
Rendite gerechnet.

Durch technische, gesellschaftliche und politische Entwicklung wurde die
Rendite über die zeit immer tiefer. Das risiko war früher sehr viel höher
als in unserer Zeit. Die Bruttorendite enthält eigentlich nur die
Leerstandsrendite.

Gesamtrendite (Performance) ist die Summe der Bruttorendite (Total der
Einkünfte über die Anlagekosten) plus die Wertänderungsrendite
(v.A. Wertsteigerung des Landes).

Netto-Cashflow-Rendite

\subsubsection{Brutto-, Netto-, Eigenkapitalrendite}
\label{sec-2-3-3}

\subsection{Chancen \& Risiken}
\label{sec-2-4}

In der Schweiz gibt es einen relativ kleinen Anteil an
Eigenheimen. Investoren wollen eine Rendite erzielen und daher sind sie wenig
interessiert an Eigentumsobjekten.

Das Interesse der Bewohner geht aber in Richtung Eigenheim. Der Markt trifft
sich nicht richtig.

In Gebieten wo die Mieten unter einem bestimmten Quadratmeterpreis sind,
können aber keine Mietobjekte mehr gebaut werden, da das Land zwar sehr
günstig ist, die Erstellungskosten aber nicht merklich kleiner sind.

Stockwerkeigentum ist wesentlich risikoreicher als Mietwohnungen. Als
Architekt muss man sich plötzlich mit einer Vielzahl an "`Bauherren"'
auseinandersetzen. Sonderwünsche für jede Partei kommen hinzu.

Normalerweise kommt eine Risikoteilung zwischen Architekten und
Investor. Dies bedeutet, dass man z.B. pro Wohnung entlohnt wird.

Der erste Schritt ist der Vorvertrag. Dies kommt auch bei einem
Bieterverfahren zum Zug. Mit dem Vorvertrag ist eine gewisse Verbindlichkeit
aber noch keine 100\% definitiven Zahlen festgehalten.

\subsection{Kosten}
\label{sec-2-5}

Kosten sind wie Kinder, man kann sie nicht vorhersehen. Die Zahlen die man
aber zu Anfang eines Projektes macht, sind immens wichtig, da das gesamte
Vorhaben darauf beruht, stimmen aber im Endeffeckt nie so wirklich.

Die Nutzungsphase beinhaltet bei weitem den Grössten Anteil der Kosten über
die Lebensdauer des Gebäudes.

Durch die Kosten die steigen, stiege eigentlich das Honorar des
Architekten. Um dem entgegenzuwirken gibt es vertragliche Lösungen, dass man
innerhalb eines bestimmten Masses genau schätzen muss. Sonst fällt eine
Strafe an.

Termine machen immer Probleme. Kosten in einer sehr frühen Phase zu schätzen
auch. Darum haben Kostenschätzungen immer eine bestimmte Genauigkeit. In
einer frühen Phase kann die Genauigkeit also bei $\pm 25\%$ liegen.

Fachplaner rechnen ihre eigenen Zahlen, allerdings fällt immer noch sehr viel
an für die Berechnung.

Es ist wichtig versteckte und ausgewiesene Reserven im Kostenvoranschlag drin
haben.

Wie ermitteln wir aber die Kosten für Objekte? Am wichtigsten ist immer Die
Grösse. Danach kommen Typologie, Baugrund, Ausbaustandard, usw. Je grösser
der Bauunternehmer, desto günstiger ist der Kubikmeterpreis, da der Markt ein
ganz anderer ist.

Umbaukosten sind extrem schwierig zu eruieren, da wir dem Bestand einen Wert
zumessen müssen.

Im Vorprojekt werden die Kosten immer konkreter, da wir mindestens im CAD
bereits flächen und Volumen messen können.

\subsection{Projektentwicklung}
\label{sec-2-6}

\subsection{Die Genossenschaft}
\label{sec-2-7}

\subsection{Wrap-Up}
\label{sec-2-8}

\section{Aufbereitung}
\label{sec-3}
% Emacs 25.3.1 (Org mode 8.2.10)
\end{document}
