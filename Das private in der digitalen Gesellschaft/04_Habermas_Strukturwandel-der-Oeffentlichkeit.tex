% Created 2018-03-08 Don 10:55
\documentclass[a4paper,ngerman,11pt]{scrartcl}
\usepackage[utf8]{inputenc}
\usepackage[T1]{fontenc}
\usepackage{fixltx2e}
\usepackage{graphicx}
\usepackage{longtable}
\usepackage{float}
\usepackage{wrapfig}
\usepackage{rotating}
\usepackage[normalem]{ulem}
\usepackage{amsmath}
\usepackage{textcomp}
\usepackage{marvosym}
\usepackage{wasysym}
\usepackage{amssymb}
\tolerance=1000
\usepackage{natbib}
\usepackage[linktocpage,pdfstartview=FitH,colorlinks,linkcolor=blue,
anchorcolor=blue,citecolor=blue,filecolor=blue,menucolor=blue,urlcolor=blue]{hyperref}
\usepackage{ngerman}
\usepackage{url}
\usepackage{breakurl}
\addtokomafont{disposition}{\rmfamily}
\subtitle{Kommentar}
\author{Christian Sangvik}
\date{\today}
\title{Strukturwandel der Öffentlichkeit\\-- Jürgen Habermas --}
\hypersetup{
  pdfkeywords={},
  pdfsubject={},
  pdfcreator={Emacs 25.3.1 (Org mode 8.2.10)}}
\begin{document}

\maketitle

\section{Autor}
\label{sec-1}

Jürgen Habermas ist ein deutscher Philosoph und Soziologe. Er wurde 1929 in
Düsseldorf als Sohn des Geschäftsführers der Industrie- und Handelskammer zu
Köln geboren. Habermas studierte zwischen 1945 und 1954, wo er sich mit
Philosophie, Geschichte, Psychologie, deutscher Literatur und Ökonomie
beschäftigte. Nach dem Studium schrieb er für diverse Zeitungen als freier
Journalist. Als Stipendiat kam er 1956 als wissenschaftlicher Assistent an das
Institut für Sozialforschung in Frankfurt. Dort kam er intensiv mit dem
Gedankengut des Marxismus und Denken von Freud in Berührung. Er engagierte
sich politisch gegen die Atomisierung und deren Auswirkungen. Ab 1961 war er
noch mitten in seinem Habilitationsprozess aber schon zum ausserordentlichen
Professor berufen. 1964 dann wurde er als Professor für Philosophie und
Soziologie and die Universität Frankfurt berufen. 1994 wurde Habermas dann
emeritiert, blieb aber aktiv und schrieb weiterhin Bücher und Kommentare. Bis
heute zählt er zu den meistzitierten Philosophen und Soziologen der
Gegenwart. Dabei gilt er als Grenzgänger zwischen Philosophie und
Sozialwissenschaften.\cite{wiki:Habermas-de}

\section{Text}
\label{sec-2}

asdf.\cite{Habermas1962}

\bibliographystyle{unsrt}
\bibliography{dasprivateinderdigitalengesellschaft}
% Emacs 25.3.1 (Org mode 8.2.10)
\end{document}
