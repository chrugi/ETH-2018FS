% Created 2018-03-10 Sam 14:52
\documentclass[a4paper,ngerman,11pt]{scrartcl}
\usepackage[utf8]{inputenc}
\usepackage[T1]{fontenc}
\usepackage{fixltx2e}
\usepackage{graphicx}
\usepackage{longtable}
\usepackage{float}
\usepackage{wrapfig}
\usepackage{rotating}
\usepackage[normalem]{ulem}
\usepackage{amsmath}
\usepackage{textcomp}
\usepackage{marvosym}
\usepackage{wasysym}
\usepackage{amssymb}
\tolerance=1000
\usepackage{natbib}
\usepackage[linktocpage,pdfstartview=FitH,colorlinks,linkcolor=blue,
anchorcolor=blue,citecolor=blue,filecolor=blue,menucolor=blue,urlcolor=blue]{hyperref}
\usepackage{ngerman}
\usepackage{url}
\usepackage{breakurl}
\addtokomafont{disposition}{\rmfamily}
\subtitle{Kommentar}
\author{Christian Sangvik}
\date{\today}
\title{Strukturwandel der Öffentlichkeit\\-- Jürgen Habermas --}
\hypersetup{
  pdfkeywords={},
  pdfsubject={},
  pdfcreator={Emacs 25.3.1 (Org mode 8.2.10)}}
\begin{document}

\maketitle
\begin{quote}
``Dann fordere ich dich zum Duell in einem herrschaftsfreien Diskurs, nach den
von Habermas aufgestellten Regeln; Wahrheit, Richtigkeit, Wahrhaftigkeit und
Verständlichkeit, und bitte darum, keine Witze über die Kombination Habermas
und Verständlichkeit zu machen.'' \nocite{Kling2014}

--- Marc-Uwe Kling
\end{quote}


\section{Autor}
\label{sec-1}

Jürgen Habermas ist ein deutscher Philosoph und Soziologe. Er wurde 1929 in
Düsseldorf als Sohn des Geschäftsführers der Industrie- und Handelskammer zu
Köln geboren. Habermas studierte zwischen 1945 und 1954, wo er sich mit
Philosophie, Geschichte, Psychologie, deutscher Literatur und Ökonomie
beschäftigte. Nach dem Studium schrieb er für diverse Zeitungen als freier
Journalist. Als Stipendiat kam er 1956 als wissenschaftlicher Assistent an das
Institut für Sozialforschung in Frankfurt. Dort kam er intensiv mit dem
Gedankengut des Marxismus und Denken von Freud in Berührung. Er engagierte
sich politisch am linken Flügel. Ab 1961, noch mitten in seinem
Habilitationsprozess, wurde Habermas bereits als ausserordentlicher, ab 1964
dann zum ordentlichen Professor für Philosophie und Soziologie an der
Universität Frankfurt berufen. 1994 wurde er emeritiert, blieb aber aktiv und
schrieb weiterhin Bücher und Kommentare. Bis heute zählt er zu den
meistzitierten Philosophen und Soziologen der Gegenwart. Dabei gilt er als
Grenzgänger zwischen Philosophie und
Sozialwissenschaften.\cite{wiki:Habermas-de}

\section{Text}
\label{sec-2}

Habermas zeigt gleich zu Beginn auf, dass die Begriffe \emph{öffentlich} und
\emph{Öffentlichkeit} einer Klärung bedürfen, da sie ``eine Mannigfaltigkeit
konkurrierender Bedeutungen'' \cite{Habermas1962} verraten, da wir sie
geschichtlich vermischen würden.

Historisch gesehen ist die Öffentlichkeit noch lange äquivalent mit dem Adel,
Hof, Fürsten und sonstigen Regenten und Klerus, ähnlich dem Römischen Vorbild.
In unserem Sprachverständnis aber bezeichnet Öffentlichkeit Dinge wie
Versnataltungen, die allen zugänglich sind. Mit dem etablieren wir einen
``Modus des Machtausgleichs'' \cite{Habermas1962} wo nicht mehr ein selektiver
Zirkel sondern prinzipiell alle teilhaben können. Mit zugänglichen Debatten
kommt ein Prinzip der Kontrolle, was die Willkür der Herrschenden erschwert.

Die Debatte und Diskussion an sich ist zugleich auch Prozess der
Selbstaufklärung der Privatleute.

Nach und nach, mit dem wichtiger werden der Städte, rückt das Verständnis der
Öffentlichkeit ab vom fürstlichen oder gar königlichen Hof. Die Stadt wird
\emph{die} Öffentlichkeit selber. Während sich der Charakter der Veranstaltungen
sich auf dem Hofe nicht gross änderte (es waren private Veranstaltungen mit
einem ausgesuchten Publikum) verstand die Bevölkerung diese Anlässe zusehends
weniger als Öffentlichkeit. Es entwickelten sich bürgerlich-intellektuelle
Zirkel die sich in zunächst in \emph{Salons} trafen und sich dort
aussprachen. Zunächst nur über Schriften und Literatur aber mehr und mehr auch
über Politik und politischer Kritik.

Diese Kreise waren zu Beginn allerdings immernoch eine erlesene Gesellschaft
der bürgerlichen Oberschicht und Stammgästen dieser Établissements. Es ging
noch darum einen Ort zu haben, wo die ``Intelligenz mit der Aristokratie
zusammentrifft'' \cite{Habermas1962}.

Nach den Salons etablierten sich auch \emph{Kaffeehäuser} welche einen zwangloseren
Zugang zu den massgeblichen Zirkeln \cite{Habermas1962} ermöglichte und so
breiteren Schichten des Mittelstandes zugänglich wurde. Es etablierte sich
eine \emph{Meinung} losgelöst von wirtschaftlichen Abhängig- und Dienstbarkeiten
gegenüber oberen Schichten. Die Wichtigkeit dieser Zusammenkünfte wird
sprachlich durch die Unterscheidung zwischen \emph{Schriften} und \emph{reden}
manifestiert. Auch hatten die Salons oft das Monopol für
Erstveröffentlichungen von Schriften und musikalischen Werken. Ein neues Opus
hatte sich zunächst vor diesem Forum zu legitimieren.\cite{Habermas1962}

Da es in Deutschland zu jener Zeit noch nicht ein Konstrukt der Stadt gab,
welches so wichtig war wie in Frankreich oder England, wo besagte Salons und
Kaffeehäuser gang  und gäbe waren, gab es noch keine Institution, welche den
repräsentativen Hof bereits hätte ablösen können. Es bildeten sich kleinere
``gelehrte Tischgesellschaften'', die von der politischen Praxis aber noch
strenger ausgeschlossen waren. Diese ebenfalls exklusiven Tischgesellschaften
waren oftmals von geheimen Charakter, da die Publizität noch immer bei der
fürstlichen Geheimkanzlei lag.\cite{Habermas1962} Solange die Öffentlichkeit
aber geheim war könne sich keine Vernunft offenbaren.

Die geheimen Gesellschaften zerfallen natürlich, mit dem Mass an Erfolg, ihre
eigene Ideologie der bürgerlichen Öffentlichkeit gegen die Reglementierung der
Obrigkeit durchzusetzten, da sie fortan nicht mehr geheim zu sein
brauchen.

Die Salons, Kaffeehäuser und Tischgesellschaften waren allesamt sehr
unterschiedlich in ihrer Art und Weise, doch war ihnen doch gemein die Tendenz
der permanenten Diskussion unter Privatleuten zu organisieren. Habermas bricht
den Charakter auf drei institutionelle Kriterien herunter:

\clearpage

\begin{itemize}
\item Es ist eine gesellschaftlicher Verkehr gefordert, der von der Gleichheit des
Status' absieht.
\item Es werden Bereiche problematisiert, die bislang nicht als fragwürdig
galten. (Dasy Allgemeine, das Profane, \ldots{})
\item Das Publikum ist prinzipiell unabgeschlossen. Alle müssen dazugehören
können.\cite{Habermas1962}
\end{itemize}

Ein Problem aus heutiger Sicht ist hier aber, dass diese Kreise doch etwas
voraussetzten, das heute kein Problem scheint, in der Zeit aber nicht
selbstverständlich war. Man musste lesen können, und man musste sich den
Zugang zur Literatur leisten können. Die Armut und der Mangel an Bildung waren
also sehrwohl Ausschliesskriterien. So war das gewünschte Zielpublikum nicht
etwa die gesamte Bevölkerung, sondern die exklusivere Schicht der Grossbürger.

Die Kunstformen der Literatur, des Theaters, der bildenden Kunst und der Musik
bekommen ihre Bedeutung als Kunst erst durch diesen Ablöseprozess vom Adel und
Hof. Das Publikum wird öffentlich. Zu Beginn allerdings noch in der Form einer
Zweiklassigkeit, die man heute noch in den Kulturgebäuden finden kann. Die
Oberschichten hoben sich vom Pöbel ab, indem sie nicht im \emph{Parterre} zu finden
waren, sondern eigene \emph{Logen} und \emph{Balkone} hatten.

Aber erst mit dieser neuen Zugänglichkeit zur Kultur und Kunst wurde auch der
Begriff des Publikums geprägt. Das Publikum war durch die Materialisierung der
Darbietung zu einem Liebhaberpublikum geworden. Jeder konnte hingehen, solange
man den Eintritt vermochte. Gleichzeitig mit der Zugänglichkeit kam natürlich
auch der Anspruch, dass jeder das Recht hat, über das erlebte zu urteilen. So
ging mit dem Liebhaberpublikum auch die Laienkritik einher.

Mit den Laienkritikern entstand auch das wesen des Berufskritikers, dem
sogenannten \emph{Kunstrichter}.\cite{Habermas1962} Die Kunstrichter waren Mandataren
und Pädagogen des Publikums zugleich. Mit der Kritik stieg auch die
Wichtigkeit der periodischen Druckpresse, welche diese Kritiken abdruckte. Sie
wurde zum Mittel der Veröffentlichung, zum Teil der Debatte und zum Wesen der
Diskussion selbst, da das neue Medium der Leserbriefe jene neue Möglichkeit
bot, dass der Leser selbst Teil wird der Drucksache.

Aber es kam auch die Frage nach der Daseinsberechtigung der Kunstkritik
auf. Denn schliesslich war die Welt nun auch jahrtausende lang ohne sie
zurechtgekommen. Die Philosophie schien zur einfachen Literatur zu verkommen,
doch ohne die Möglichkeit der Auseinandersetzung mit der Literatur und Kunst,
können sich die Leute auch nicht selber aufklären und zu einem Verständnis der
Philosophie gelangen.

Ab der zweiten Hälfte des 18. Jahrhunderts können wir von einem Verständnis
von Öffentlichkeit sprechen, das in den Grundzügen unserem heutigen
entspricht. Die Öffentlichkeit sind nicht mehr wenige Akteure, sondern
prinzipiell jeder. Mit diesem Wandel wandelte sich nun aber auch die
Familie. Sind Begriffe wie \emph{öffentlich} und \emph{privat} zu Beginn eigentlich
nicht zu gebrauchen, treten sie jedoch bald in den Jargon des Volkes.

Auch die Architektur wandelte sich mit dem einhergehend. Die diendenden Räume
werden auf das Minimum beschränkt, und statt dem überal Repräsentativen tritt
nun eine Unterscheidung von privat und öffentlich. Familienräume weichen
Empfangszimmern und Privaträumen für die einzelnen Familienmitglieder und die
Wohnhalle weicht dem Wohnzimmer. Als repräsentatives Relikt bleibt der
Salon. Der Salon dient auch nicht dem Hause sondern der Gesellschaft. So geht
die Grenze zwischen öffentlich und privat durch das eigene Haus. Eine
Vorstellung die mit unserer heutigen Vorstellung nicht einher geht, da wir uns
an \emph{die eigenen vier Wände} gewöhnt haben. Wir möchten auf unserem Grund keine
Öffentlichkeit, sondern geniessen die Abgeschiedenheit. Wir setzen im Privaten
auf den intimen Rahmen der kleinen Familie.

Mit der Verlagerung in Richtung der Intimisphäre verändert sich auch die
Literatur. Von repräsentativ wechselt sie auch richtung Gefühlsorientiert, bis
dahin, dass die vorherrschende literarische Form ende des Jahrhunderts der
Brief ist. Und mit der Öffnung zur Gefühlswelt tritt auch die Fiktion in die
Literatur ein. So entsteht die Belletristik, und mit ihr auch Buchclubs,
Lesezirkel und auch erste Bibliotheken.\cite{Habermas1962}

\bibliographystyle{unsrt}
\bibliography{dasprivateinderdigitalengesellschaft}
% Emacs 25.3.1 (Org mode 8.2.10)
\end{document}
