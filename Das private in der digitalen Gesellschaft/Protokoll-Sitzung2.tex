% Created 2018-03-05 Mon 01:40
\documentclass[a4paper,ngerman,11pt]{scrartcl}
\usepackage[utf8]{inputenc}
\usepackage[T1]{fontenc}
\usepackage{fixltx2e}
\usepackage{graphicx}
\usepackage{longtable}
\usepackage{float}
\usepackage{wrapfig}
\usepackage{rotating}
\usepackage[normalem]{ulem}
\usepackage{amsmath}
\usepackage{textcomp}
\usepackage{marvosym}
\usepackage{wasysym}
\usepackage{amssymb}
\tolerance=1000
\usepackage{natbib}
\usepackage[linktocpage,pdfstartview=FitH,colorlinks,linkcolor=blue,
anchorcolor=blue,citecolor=blue,filecolor=blue,menucolor=blue,urlcolor=blue]{hyperref}
\usepackage{ngerman}
\addtokomafont{disposition}{\rmfamily}
\subtitle{Protokoll der Sitzung vom 27.2.2018}
\author{Christian Sangvik}
\date{4. März 2018}
\title{Post-Privacy}
\hypersetup{
  pdfkeywords={},
  pdfsubject={},
  pdfcreator={Emacs 25.3.1 (Org mode 8.2.10)}}
\begin{document}

\maketitle

\section{Generelles}
\label{sec-1}

Wir haben in der zweiten Sitzung den Text ``Post-Privacy'' von Christian Heller
diskutiert.\cite{Heller2011}

Beim Lesen des Textes ist aufgefallen, dass Heller den Begriff der
Privatsphäre gar nie ausführt, sondern einfach im Sinne einer Prämisse davon
ausgeht, dass es eben jene nicht mehr gibt und ein eigentlicher Datenschutz
nicht mehr möglich sei. Dies ist nicht der einzige Punkt,
wo es scheint, dass er seine eigenen Prämissen nicht überprüft.

Abgesehen von nicht diskutierten Behauptungen zeigt Heller aber auf einer sehr
bildlichen und metaphorischen Weise, wie er uns Menschen in Interaktion mit
dem Internet sieht.

Wir haben zum diskutieren eine Grundstruktur basierend auf vier Fragen
erarbeitet:

\begin{itemize}
\item Warum verlegen die Leute ihr Leben ins Internet?
\item Wie funktioniert das Internet in technischer Sicht?
\item Was passiert mit dem Menschen im digitalen Zeitalter?
\item Was tun ausgehend von der jetzigen Situation?
\end{itemize}

Bemerkenswert an unserer Diskussion war ausserdem, dass wir, selbst nachdem
wir uns zu beginn schon klar gemacht haben, dass der Autor die Privatsphäre
nicht diskutiert hat, dies auch nicht getan haben. Ist dies ein Symptom einer
Diskussion über das Internet und das Private?

\section{Leben im Internet}
\label{sec-2}

Warum der Mensch sich dem Internet preisgibt lässt sich nicht sagen. Aber wir
können mehrere Beweggründe vermuten.

Es scheint Teil einer jeden Gesellschaft zu sein, dass sich die Menschen in
ihr bisweilen verstellen, aus dem Alltagstrott ausbrechen, oder in einer
Ausnahmesituation, wo sie sich zusammenrotten müssen, hinter eine andere
Fassade begeben wollen. Wir müssen quasi am Netz teilnehmen, um nicht den
Anschluss an die Gesellschaft zu verlieren. Und wer teilnimmt wird
gleichzeitig zum Teil des Netzes.

Es scheint bisweilen sogar notwendig, dass sich der Mensch verwandelt, damit
auch andere Seiten seiner selbst zum Tragen kommen, welche er normalerweise
nicht ausleben kann. Die Verwandlung selber kann auch als ein Ritual gelesen
werden.

Ich denke, dass wir uns einigermassen einig sind, dass wir uns mit der
Teilnahme nicht notwendigerweise selber aufgeben, sondern dass wir nur eine
Maske aufsetzen, wie auch am Karnevall, im Beruf oder bei anderen Formen der
Selbstinszenierung. Der Unterschied scheint jedoch darin zu bestehen, dass im
Internet jeden Tag Karnevall ist.

Ein durchaus nicht zu vergessender Punkt für das Netz ist dessen Einfachheit
und Effizienz. Wir können ungeachtet grosser Distanzen in Echtzeit
Informationen austauschen, was unseren Alltag mittlerweile massgebend
bestimmt.

\section{Das Internet}
\label{sec-3}

Heller beschreibt das Internet als ein organisches Wesen, eine Art Monster,
das aktiv agiert und nach Daten hungert, welche der Mensch bereitwillig
füttert. Und das Monster ist sehr hungrig, da es ständig mehr lernen
will. Sein Wesen scheint das mehr lernen zu sein, da es so konzipiert ist.

Allgemein wird die technische Funktionsweise des Internets im Text nicht
diskutiert, sondern ein Schauermärchen aus Bildern entwickelt. Das Monster hat
hunger, ein Bild, welches forciert wiederholt wird zur Verstärkung seiner
Wirkung.

In seinem Wissensdurst und durch die Art, wie das Netz als Schnittstelle des
Austausches jedes einzelnen Computers, und damit von den Menschen die sich
hinter dem Schirm befinden, steht, wird das Netz selber unkontrollierbar und
legt sich so flächendeckend über den gesamten Planeten. Das Netz ist nicht
Teil der Kommunikation, es ist vielmehr die Kommunikation selbst. Wir möchten
eingentlich über das Netzwerk mit anderen kommunizieren, kommunizieren aber
letzten Endes mit dem Netz selbst.

Das Internet ist eine wechselseitige Beziehung nach kommerzieller Art. Wir
geben Daten hinein, und bekommen dafür andere Daten zurück. Allerdings können
wir nur schwerlich den Preis ausmachen, da wir nicht genau wissen, welche
Daten denn eigentlich von uns genommen werden, und auch nicht, wozu diese
Daten denn letzten Endes gebraucht werden. Es stellt sich denn auch die Frage;
Was ist Information, was sind Daten? Klar ist nur, dass das Internet eine Art
Basar für Daten ist. Jedem ist auch klar, dass dieser Basar auch überwacht
ist. Doch wer überwacht ihn, und in welchem Masse? Müssen wir uns denn
überhaupt vor einer Überwachung fürchten, oder mag sie uns sogar dienen?

Eine andere Eigenheit des Netzes ist dessen Speicherfunktion. Was einmal ins
Netz eingespeist wurde, geht niemals wieder weg. Daten, die ihren Weg ins Netz
gefunden haben, können nicht mehr gelöscht werden, da sie bereits mit der
Eingabe x-Fach kopiert und eingelagert wurden. Das Internet ist nicht eine
Oberfläche, die sich den Menschen die mit ihm interagieren zeigt, sondern eine
Art Maschinenraum, der ständig arbeitet, Daten verschiebt und kopiert, neues
generiert und niemals schläft. Wir brauchen die Daten des Internets nicht
einmal selber zu generieren, sondern alleine die Metadaten, die durch die
Nutzung anfallen sind neue verwendbare Daten.

\section{Die Zukunft}
\label{sec-4}

Das Internet ist aus einer guten Idee heraus entstanden, einfach und von
überal aus sich austauschen zu können. Es ist aber wichtig, dass es ein können
und kein müssen ist. In der heutigen Zeit aber ist es zu einem Zwang
geworden. Viele Alltagshandlungen sind ohne Internet nicht mehr denkbar,
selbst wenn das Netzwerk für einiges verborgen im Hintergrund bleibt.

Es ist also wahrscheinlich nicht der richtige Weg, gegen das Internet
ankämpfen zu wollen. Wir sollten uns aufklärerisch auseinandersetzen, was
diese neue Lebensweise für uns bedeutet, und wir müssen uns positionieren, wie
wir damit umgehen wollen.

Viele wünschen sich ein Regelwerk, welches Gefahren und Risiken minimiert. Ein
solches Regelwerk ist aber nur schwerlich umzusetzen, da das Internet nicht
mehr territorial funktioniert, wie dies klassische Gesetzgebungen
tun. Ausserdem, selbst wenn wir eine Regelung hätten, wer würde diese
forcieren? Das Vertrauen der Menschen in irgendwelche Instanzen, seien sie
national, international oder gar privat, scheint gebrochen oder mindestens
belastet. Wir hören immer wieder von neuen Rechtsüberschreitungen im grossen
Stil, und dies auch von Staatswegen.

Ein anderes Problem ist die Frage, ob wir uns denn, selbst wenn wir so
gewissenhaft mit der Ressource Internet umgehen, wie wir nur können, überhaupt
selber schützen können, oder sind wir bereits Opfer von der Nutzung anderer,
welche dann Rückschlüsse auf uns zulassen? Können wir das Internet torpedieren
mit gezielten Angriffen und Fehlinformationen? Ich denke nicht. Aber eine
Bewegung in die richtige Richtung wäre in meinen Augen, wenn man das Konzept
des Internets wieder vermehrt wie in dessen Anfängen lesen und auch leben
würden: Ein \emph{dezentrales} Netzwerk, an dem jeder teilhaben \emph{kann}. Wir müssen
mit Bedacht und Bewusstsein damit umgehen.

\bibliography{dasprivateinderdigitalengesellschaft}
\bibliographystyle{unsrt}
% Emacs 25.3.1 (Org mode 8.2.10)
\end{document}
