% Created 2018-03-26 Mon 00:14
\documentclass[a4paper,ngerman,11pt]{scrartcl}
\usepackage[utf8]{inputenc}
\usepackage[T1]{fontenc}
\usepackage{fixltx2e}
\usepackage{graphicx}
\usepackage{longtable}
\usepackage{float}
\usepackage{wrapfig}
\usepackage{rotating}
\usepackage[normalem]{ulem}
\usepackage{amsmath}
\usepackage{textcomp}
\usepackage{marvosym}
\usepackage{wasysym}
\usepackage{amssymb}
\tolerance=1000
\usepackage{natbib}
\usepackage[linktocpage,pdfstartview=FitH,colorlinks,linkcolor=black,
anchorcolor=black,citecolor=black,filecolor=black,menucolor=black,urlcolor=black]{hyperref}
\usepackage{ngerman}
\usepackage{url}
\usepackage{breakurl}
\addtokomafont{disposition}{\rmfamily}
\subtitle{Kommentar}
\author{Christian Sangvik}
\date{25. März 2018}
\title{Soziologie}
\hypersetup{
  pdfkeywords={},
  pdfsubject={},
  pdfcreator={Emacs 25.3.1 (Org mode 8.2.10)}}
\begin{document}

\maketitle

\section{Autor}
\label{sec-1}

Georg Simmel war ein deutscher Philosoph und Soziologe und begründete die
\emph{formale Soziologie} und \emph{Konfliktsoziologie}. Er wurde zu einer
preisgekrönten Schrift über Kants Materiebegriff promoviert, nachdem seine
eigentlich erste Dissertation über Musikethnologie abgelent wurde. Auch seine
Hablilitationsschrift handelte über Kant, diesmal von dessen Lehre von Raum
und Zeit. Allgemein stand Simmel in der Tradition des Neukantismus und der
Lebensphilosophie.

Sein Haus in Berlin wurde zu einem Ort des geistigen Austausches, wo sich
Leute wie Rainer Maria Rilke Max Weber und viele mehr trafen. Da seine Familie
eine jüdische Vergangenheit hatte, war es für Simmel im Deutschland um die
Jahrhundertwende nicht einfach eine gebürende Anerkennung für seine Arbeit zu
erhalten. Heute aber wird er von vielen als Begründer der deutschen Soziologie
angesehen.\cite{wiki:Simmel-de}

\section{Text}
\label{sec-2}

Simmel beschreibt im Textauszug Beweggründe für das Bilden von
Geheimgesellschaften sowie die soziologisch-gesellschaftliche Wirkungsweise
einer solchen. Die treibende Kraft für den Erhalt einer solchen geheimen
Gesellschaft ist das Vertrauen der einzelnen Menschen die Teil einer solchen
Gemeinschaft sind untereinander. Dafür bietet das geheime Kollektiv durch
seine verborgene Natur Schutz für die einzelnen Menschen. Während die existenz
eines Einzelnen nicht wirklich geheim sein kann, können doch Zusammenkünfte
und Gemeinschaften eine vollkommen verborgene Existenz haben.\cite{Simmel1992}

Eine Schwachstelle bildet hier der Umstand, dass wenn auch nur marginales oder
gar Gerüchte über eine solche geheime Gesellschaft ans Licht kommen, diese
weitere Angriffsmöglichkeit für das Enttarnen jener bietet. Somit schwebt bei
einem Geheimbund immer auch die subtile Gefahr des Entdecktwerdens mit. Simmel
schreibt wörtlich ``daß man mit Recht sagt, ein Geheimnis, um das Zwei wissen,
sei keines mehr''.\cite{Simmel1992} Dies ist auf zweierlei Arten
problematisch. \emph{Erstens} muss das Geheimnis von mindestens zwei Leuten gewusst
werden, wenn sie sich zusammentun unter dem Inhalt des Geheimnisses. Aber dies
macht \emph{zweitens} die Sicherheit des Einzelnen vom Anderen abhängig. Man legt
quasi seine eigene Sicherheit in die Hände des Anderen. Und wenn eine
Gesellschaft geheim sein muss, dann ist dies wohl aus dem Grund, dass
Praktiken oder Wissen, welches die Gesellschaft gemein haben wohl eher
prekärer Natur sind, und so die Sicherheit, wenn der Bund auffliegt garantiert
nicht gewärleistet werden kann.

Die zugehörigen müssen sich zwangsläufig gegenseitig schützen um nicht
entdeckt zu werden, was aber schwierig ist, selbst wenn keine Sabotage
vorliegt, kein Maulwurf eingeschleust ist und dergleichen, da schon eine
kleine Unachtsamkeit in einem willkürlichen Moment ausreichend sein kann um
das Geheimnis auffliegen zu lassen. Im Laufe der Zeit haben verschiedenste
Gesellschaften ganz unterschiedliche Wege gefunden um die einzelnen Mitglieder
zu sensibilisieren, oder die Verschwiegenheit zu indoktrinieren.\cite{Simmel1992}

Drei Bemerkungen oder Fragen, die mir beim Lesen des Textes aufgekommen sind
möchte ich hier hervorheben:

\begin{itemize}
\item Simmel beschreibt, dass die Schriftlichkeit an sich in ihrem Wesen der
Geheimhaltung widerstrebt.\cite{Simmel1992} Dies finde ich sehr einleuchtend,
doch wissen wir auch, dass über die Geschichte viele Methoden gefunden
wurden mit diesem Problem umzugehen. Der Text wurde entweder versteckt, so
dass er für nicht Eingeweihte nicht auffind- oder lesbar war, oder aber er
wurde verschlüsselt, so dass er für nicht Engeweihte selbst wenn er gefunden
wurde keinen Sinn ergab. Dies sind auch die Schutzmechanismen, die wir im
digitalen haben.
\item Durch die Schriftlosigkeit ist die Abhängigkeit von einzelnen Personen sehr
gross. Dies gibt einem gewissen Individuum in der Gesellschaft einen hohen
Stellenwert und Anerkennung. Die Gesellschaft ist abhängig von gewissen
Schlüsselmitgliedern. Auf der anderen Seite sind diese aber auch im gleichen
Masse abhängig von der Gesellschaft selber. Sie können mit ihrem
Sonderwissen keinen Druck aufsetzen und sich übervorteilen, da ausserhalb
der Gesellschaft dieses Können oder Wissen nichts wert ist. So kann sich
kein allzu grosser Hebel der Ungerechtigkeit bilden. Alle sind letzten Endes
von allen anderen abhängig.
\item Zuletzt habe ich mich gefragt, was wohl der Wert von Information ist. Wert
meine ich hier in den Bedeutungen Gegenwert wie auch Sinn. Wenn ein
Geheimnis brisant ist, kommen wir sicherlich schnell überein, dass dies eine
wertvolle Information ist, die Schutz braucht. Wie sieht dies aber aus bei
einer Banalität? Was ist der Wert einer Banalität? Man kann zwar
argumentieren, dass sich niemand wirklich dafür interessieren mag, aber das
bestehen eines Geheimnisses ist wie wir gelernt haben auch abhängig von ganz
kleinen aussagen, die verraten könnten. Inwiefern gilt dies auch für die
digitale Gesellschaft?
\end{itemize}

\bibliographystyle{unsrt}
\bibliography{dasprivateinderdigitalengesellschaft}
% Emacs 25.3.1 (Org mode 8.2.10)
\end{document}
