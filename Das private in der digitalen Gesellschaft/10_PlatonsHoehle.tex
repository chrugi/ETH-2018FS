% Created 2018-05-08 Di 17:12
\documentclass[a4paper,ngerman,11pt]{scrartcl}
\usepackage[utf8]{inputenc}
\usepackage[T1]{fontenc}
\usepackage{fixltx2e}
\usepackage{graphicx}
\usepackage{longtable}
\usepackage{float}
\usepackage{wrapfig}
\usepackage{rotating}
\usepackage[normalem]{ulem}
\usepackage{amsmath}
\usepackage{textcomp}
\usepackage{marvosym}
\usepackage{wasysym}
\usepackage{amssymb}
\tolerance=1000
\usepackage{natbib}
\usepackage[linktocpage,pdfstartview=FitH,colorlinks,linkcolor=black,
anchorcolor=black,citecolor=black,filecolor=black,menucolor=black,urlcolor=black]{hyperref}
\usepackage{ngerman}
\usepackage{url}
\usepackage{breakurl}
\addtokomafont{disposition}{\rmfamily}
\subtitle{Kommentar}
\setcounter{secnumdepth}{0}
\author{Christian Sangvik}
\date{8. Mai 2018}
\title{In Platons Höhle}
\hypersetup{
  pdfkeywords={},
  pdfsubject={},
  pdfcreator={Emacs 25.3.1 (Org mode 8.2.10)}}
\begin{document}

\maketitle

\section*{Autor}
\label{sec-1}

Susan Sontag (* 16. Januar 1933 in New York City, † 28. Dezember 2004 in New
York City) war eine amerikanische Schriftstellerin und Regisseurin. Sie setzte
sich für Menschenrechte ein und kritisierte die gesellschaftlichen
Verhältnisse sowie die Regierung der Vereinigten Staaten.\cite{wiki:Sontag-de}

Sie studierte an der Berkeley Universität, wechselte aber dann nach Chicago,
wo sie Literatur, Theologie und Philosophie.\cite{wiki:Sontag-de}

\section*{Text}
\label{sec-2}

Sontag beschreibt in ihrem Text verschiedene Faktoren von unserem Umgang mit
dem Medium der Fotografie. Im Einstieg erklärt sie, wie wir uns selber nach
einem neuen visuellen Code erziehen, und so unsere Wahrnehmung an das
anpassen, was wir im Alltag auf Fotografien sehen.\cite{Sontag2010} Wir scheinen
unterbewusst Besitz von dem Sujet ergreiffen zu wollen, indem wir die
Erfahrung über das Ansehen des Bildes in uns aufnehmen.

Dass ein Foto niemals eine objektive und neutrale Darstellung von etwas sein
kann ist uns vielleicht klar, doch sind wir doch Opfer des Umstandes, dass wir
das gesehene sehr leicht und ohne zu reflektieren aufnehmen können, und somit
wahrscheinlich häufig zu unkritisch sind. Jede Fotografie ist das Produkt
einer mehr oder minder bewussten Inszenierung eines Fotografen. Und einige
betreiben einen riesigen Aufwand, um am Schluss genau das Bild, das sie sich
schon vorher überlegt haben, transportieren zu können.

Fotos dienen im allgemeinen dem Beweise irgendeines Ereignisses, der
Überwachung, der Rechtfertigung, des darstellenden Zeigens, des Notierens und
vor allem des Bewahrens eines gerahmten Ausschnittes in der Zeit. Dies ist für
mich kein neues Wissen. Aber, dass die Aufnahme eines Fotos eine
passiv-aggressive Handlung darstellt \cite{Sontag2010} ist für mich eine
Überlegung, die ich noch nicht gemacht habe. Das Foto an sich bestrebe, "`so
viele Motive wie nur möglich einzufangen"'.\cite{Sontag2010} Sie ist zu einem
gesellschaftlichen Ritus geworden, und dadurch zu einem "`Abwehrmittel gegen
Ängste und ein Instrument der Macht"'.

Ein Teil dieser Macht sei jene des Voyeurs. Durch das bewusste agieren in
einer von unsicherheit geprägten Welt und Gesellschaft \cite{Sontag2010} kommt
man durch die Handlung des Fotografierens in eine Rolle des Akteurs und
verbleibt nicht als ausgelieferter. Dies ist aber sicherlich durch andere
Aktivitäten als das Fotografieren auch möglich. Eine viel wichtigere Facette
der Macht könnte also jene sein, dass die Fotografie den Maßstab aus den
Ereignissen die festgehalten werden nimmt. Jedes Foto ist am Schluss in etwa
der gleichen Form und mit der gleichen Wichtigkeit da, egal ob dies
"`jugendliche Mätzchen, Kolonialkriege [oder] Wintersport"' \cite{Sontag2010}
ist. Sie werden abgelegt, und eventuell abgedruckt und bleiben als flache
Repräsentation des geschehenen. Durch den Akt des Fotografierens selber
entsteht selber schon ein Ereignis, welches "`immer mehr gebieterische Rechte
verleiht: sich einzumischen, in das, was geschieht, es zu ursupieren oder aber
zu ignorieren"' \cite{Sontag2010}. Das Fotografieren ist ein "`Akt der
Nicht-Einmischung"' \cite{Sontag2010}, wo man von den moralischen Grundsätzen,
die wir vermutlich aus der Kinderstube mitgenommen haben absehen, und uns
bewusst dafür entscheiden, lieber ein Bild aufzunehmen, als unter Umständen
helfend oder rettend einzugreifen.

Ich muss gestehen, dass ich mit dem Teil über die Macht der Fotografie ein
wenig überfordert war, denn ich sehe das ganze nicht in dem Masse so. Mit der
Fotografie habe ich sicherlich die Macht eine Stimmung zu beeinflussen, oder
einen Anstoss für eine Bewegung zu geben, doch würde ich dies nicht dermassen
überdramatisieren. Für mich ist der aussschlaggebendste Punkt eines
unüberlegten Umganges, nicht jener eines Missbrauches einer Macht, sondern die
Tatsache, wie omnipräsent die Fotografie ist, und die Abstumpfung die damit
einher geht.

Die Fotografie ist ein Medium, eine Sprache, die von allen verstanden wird,
die sehr direkt in unser Hirn geht, und auch unterbewusst wirkt, Emotionen und
Reaktionen auslöst. Der Überkonsum führt aber zu einer Abstumpfung.

Den Bezug zum Seminar sehe ich an verschiedenen Punkten. Zum einen denjenigen,
der in Richtung der schon öfters diskutierten Masken geht. Wie stelle ich mich
dar? Wie möchte ich wahrgenommen werden? Ich gestalte bewusst ein Abbild von
mir nach aussen, das nicht mit dem inneren Bild von mir selbst korrelieren
muss. Der andere Punkt ist die Frage, wozu ich denn überhaupt fotografiere?
Sind die Bilder, die ich anfertige dafür gedacht, dass ich mich in einer
ruhigen Minute nochmals zusammenfassend mit etwas bereits erlebtem
auseinandersetzen kann und dies revue passieren lassen kann, oder fertige ich
Bilder an, um die Maske zu transportieren?

\bibliographystyle{unsrt}
\bibliography{dasprivateinderdigitalengesellschaft}
% Emacs 25.3.1 (Org mode 8.2.10)
\end{document}
