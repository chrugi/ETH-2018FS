% Created 2018-03-17 Sam 20:49
\documentclass[a4paper,ngerman,11pt]{scrartcl}
\usepackage[utf8]{inputenc}
\usepackage[T1]{fontenc}
\usepackage{fixltx2e}
\usepackage{graphicx}
\usepackage{longtable}
\usepackage{float}
\usepackage{wrapfig}
\usepackage{rotating}
\usepackage[normalem]{ulem}
\usepackage{amsmath}
\usepackage{textcomp}
\usepackage{marvosym}
\usepackage{wasysym}
\usepackage{amssymb}
\tolerance=1000
\usepackage{natbib}
\usepackage[linktocpage,pdfstartview=FitH,colorlinks,linkcolor=blue,
anchorcolor=blue,citecolor=blue,filecolor=blue,menucolor=blue,urlcolor=blue]{hyperref}
\usepackage{ngerman}
\usepackage{url}
\usepackage{breakurl}
\addtokomafont{disposition}{\rmfamily}
\subtitle{Kommentar}
\author{Christian Sangvik}
\date{\today}
\title{Grundfragen des Datenschutzes}
\hypersetup{
  pdfkeywords={},
  pdfsubject={},
  pdfcreator={Emacs 25.3.1 (Org mode 8.2.10)}}
\begin{document}

\maketitle

\section{Autor}
\label{sec-1}

Wilhelm Steinmüller (* 29.5.1934, † 1.2.2013) war ein deutscher Jurist und
Psychotherapeut. Er studierte Rechtswissenschaften, evangelische Theologie,
Informatik und Volkswirtschaft. Später war er tätig als Professor für
Kirchenrecht an der Uni Regensburg und ab 1982 Professor für angewandte
Informatik an der Uni Bremen. 1990 legte er noch eine Ausbildung als
Psychotherapeut ab.\cite{wiki:Steinmueller-de} Wie es seine Bildung vermuten
lässt, hat er querbeet durch alle Fachrichtungen publiziert. Bemerkenswert
hierbei auch für unser Seminar ist sein Lehrbuch für angewandte Informatik
\emph{Informationstechnologie und Gesellschaft}, das 1993
erschien.\cite{wiki:Steinmueller-de}

Alfred Bussey (* 10.11.1915, † 6.11.1987) war ein schweizer Nationalrat der
SP.\cite{wiki:Bussey-de} Abgesehen von seiner politischen Laufbahn habe ich
keine weiteren Angaben über ihn in Erfahrung bringen können.

\section{Text}
\label{sec-2}

In Busseys Postulat können wir eine beginnende Skepsis gegenüber den
Datenverarbeitungsmöglichkeiten durch Computer lesen. Die Gefahr kommt seines
Erachtens aus dem Zusammenführen von Verschiedenen Informationen, die nicht
zwingend miteinander zu tun haben, an zentrale Stellen, wo sie auch von
anderenorts abrufbar werden. Also grundsätzlich sind die Probleme hinsichtlich
des Persönlichkeitsschutzes der EDV die Geschwindigkeit und die Vernetzung.
Zwar beschreibt er in einer positiven Stimmung und mit wörtlicher Ausführung
des Vertrauens gegenüber den Staatsorganen (``[\ldots{}] dass unsere Verwalgungen
die so gesammelten Auskünfte mit der erforderlichen Zurückhaltung behandeln
werden[\ldots{}]'' \cite{Computer1972}) doch sind offensichtliche Bedenken über die
selben Möglichkeiten von privaten (und somit unkontrollierbaren) Instanzen zu
lesen (``[\ldots{}] doch gilt dies nicht ohne weiteres für die privaten
Unternehmen.'' \cite{Computer1972}). Er stellt den Antrag eine mögliche neue
Gesetzgebung zu prüfen, die ``den Bürger und seine Privatsphäre gegen die
missbräuchliche Verwendung der Computer schützen'' \cite{Computer1972} aber auch
``eine normale Entwicklung der Verwendung von Computern ermöglichen''
\cite{Computer1972}.

Eine solche Gesetzgebung, die den Persönlichkeitsschutz abdeckt scheint aber
sonderlich schwierig, wie wir dem \emph{Gutachten} von \emph{Prof. Dr. W. Steinmüller et
al.} entnehmen können. Das grundlegende Problem beim Schaffen eines solchen
Gesetzes ist die unpräzise Natur des Begriffes der Privatsphäre. Dieses
Problem ist, da es zum Kerninhalt unseres Seminares gehört, teil jeder
Diskussion und wir tun uns bereits im kleinen Rahmen mit sehr wenigen Personen
erstaunlich schwer, einen gemeinsamen Nenner zu finden, worum es sich
eigentlich beim Begriff \emph{Privatsphäre} handelt. Dass sich dieses Problem, wenn
es eine gesamte Bevölkerung betreffen soll, noch ungemein steigert scheint
logisch.

Im wesentlichen Beschreibt Steinmüller in seinem Gutachten, dass die
Privatsphäre mit den aktuellen Gesetzen (von 1971 aber ich denke dass dies
auch für die heutige Zeit gilt) nicht zu schützen ist.\cite{Datenschutz1971} Es
müssten also tatsächlich neue Gesetzesschriften geschaffen werden für diesen
Zweck. Aber das Problem ist in der Unfassbarkeit des Begriffes
\emph{Privatsphäre}. Ein Versuch die Privatsphäre positiv zu definieren sieht er
als gescheitert und gar unmöglich an (``Eine positive Inhaltsbestimmung ist
wegen der Relativität der ``Privatsphäre'' unmöglich.''
\cite{Datenschutz1971}). Diese Relativität hängt zum einen damit zu sammen, dass
sich die Ansicht über den Inhalt der Privatsphäre über die Zeit ändern, und
auch von Person zu Person, und darüber hinaus auch von wem Informationen
offenbar werden sollen, abhängig ist.\cite{Datenschutz1971} Ich denke dies genau
spiegelt die Natur der Problematik besonders gut, wie wir auch in unserem
Seminar jeweils zu differenzieren versuchen, welche unserer persönlichen
Informationen nun privat sind, und welche nicht, und uns nicht einig werden
darüber.

Als erschreckend einleuchtenden Punkt, der sich von der Absicht einer jeden
Rechtssprechung abheben muss führt Steinmüller auch das Argument ein, dass
``ihre [Privatsphäre] Schutzwürdigkeit in der Regel erst beurteilt werden
kann, wenn sie bereits verletzt ist'' \cite{Datenschutz1971}, und dass demnach
diese Beurteilung einiger Willkür seitens der Judikativen unterworfen ist.

Nach dem Aufzeigen von Problemen bei einer neuen Rechtsimplementierung durch
die Legislative oder durch die Judikative, welche jeweils andere Probleme mit
sich bringen, aber keine geeigneter scheint, als die andere, führt Steinmüller
auch andere Begriffe wie \emph{Privatheit}, \emph{Erheblichkeit} \cite{Datenschutz1971}
und \emph{Identifizierbarkeit} als mögliche Substitute des Begriffes \emph{Privatsphäre}
ins Feld, welche aber sicherlich niemals hinreichend eine Definition die
rechtlich tauglich wäre zulassen. Er kommt zum Schluss, dass ``das Reizwort
``Privatsphäre'' [\ldots{}] darum [Relativität, Unbestimmtheit und rechtliche
unbrauchbarkeit] aus der wissenschaftlichen Diskussion des Datenschutzrechts
auszuscheiden'' \cite{Datenschutz1971} sei. Ein neuer Ansatz sei ``in der
Anknüpfung an die Information und ihre Verarbeitung'' \cite{Datenschutz1971}.

Ich für meinen Teil fand diesen Text sehr erleuchtend, und ich bin mir sicher
nun ein wenig besser zu verstehen, warum Datenschutz aus rechtlicher Sicht
nicht einfach sein kann. Auch bringt es mich zum Überlegen, meine generellen
Forderungen für mehr Datenschutzgesetze auch für mich selber besser
ausformulieren zu sollen.

\bibliographystyle{unsrt}
\bibliography{dasprivateinderdigitalengesellschaft}
% Emacs 25.3.1 (Org mode 8.2.10)
\end{document}
