% Created 2018-03-05 Mon 11:25
\documentclass[a4paper,ngerman,11pt]{scrartcl}
\usepackage[utf8]{inputenc}
\usepackage[T1]{fontenc}
\usepackage{fixltx2e}
\usepackage{graphicx}
\usepackage{longtable}
\usepackage{float}
\usepackage{wrapfig}
\usepackage{rotating}
\usepackage[normalem]{ulem}
\usepackage{amsmath}
\usepackage{textcomp}
\usepackage{marvosym}
\usepackage{wasysym}
\usepackage{amssymb}
\tolerance=1000
\usepackage{natbib}
\usepackage[linktocpage,pdfstartview=FitH,colorlinks,linkcolor=blue,
anchorcolor=blue,citecolor=blue,filecolor=blue,menucolor=blue,urlcolor=blue]{hyperref}
\usepackage{ngerman}
\usepackage{url}
\addtokomafont{disposition}{\rmfamily}
\subtitle{Kommentar}
\author{Christian Sangvik}
\date{4. März 2018}
\title{Das Private im Öffentlichen Raum\\-- Paul Veyne --}
\hypersetup{
  pdfkeywords={},
  pdfsubject={},
  pdfcreator={Emacs 25.3.1 (Org mode 8.2.10)}}
\begin{document}

\maketitle

\section{Autor}
\label{sec-1}

Paul Veyne ist ein französischer (Alt-)Historiker, spezialisiert auf die
römische Geschichte. Er wurde am 13. Juni 1930 in Aix-en-Provence geboren,
studierte an der \emph{École normale supérieure} und doziert seit 1976 als
Professor für römische Geschichte am \emph{Collège de France} in Paris. Persönlich
stand er Michel Foucault vor dessen Tod sehr nahe.\cite{wiki:PaulVeyne-de}

\section{Text}
\label{sec-2}

asdf.\cite{Veyne1989} asasdfasdfasdf.\cite{Veyne1989}

\bibliographystyle{unsrt}
\bibliography{dasprivateinderdigitalengesellschaft}
% Emacs 25.3.1 (Org mode 8.2.10)
\end{document}
