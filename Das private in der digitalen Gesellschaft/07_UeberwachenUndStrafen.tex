% Created 2018-04-10 Die 15:11
\documentclass[a4paper,ngerman,11pt]{scrartcl}
\usepackage[utf8]{inputenc}
\usepackage[T1]{fontenc}
\usepackage{fixltx2e}
\usepackage{graphicx}
\usepackage{longtable}
\usepackage{float}
\usepackage{wrapfig}
\usepackage{rotating}
\usepackage[normalem]{ulem}
\usepackage{amsmath}
\usepackage{textcomp}
\usepackage{marvosym}
\usepackage{wasysym}
\usepackage{amssymb}
\tolerance=1000
\usepackage{natbib}
\usepackage[linktocpage,pdfstartview=FitH,colorlinks,linkcolor=black,
anchorcolor=black,citecolor=black,filecolor=black,menucolor=black,urlcolor=black]{hyperref}
\usepackage{ngerman}
\usepackage{url}
\usepackage{breakurl}
\addtokomafont{disposition}{\rmfamily}
\subtitle{Kommentar}
\setcounter{secnumdepth}{0}
\author{Christian Sangvik}
\date{4. April 2018}
\title{Überwachen und Strafen}
\hypersetup{
  pdfkeywords={},
  pdfsubject={},
  pdfcreator={Emacs 25.3.1 (Org mode 8.2.10)}}
\begin{document}

\maketitle

\section*{Autor}
\label{sec-1}

Michel Foucault (* 15.10.1926, † 25.06.1984) war ein französicher Philosoph,
Psychologe und Soziologe. Er gilt als Begründer der
Diskursanalyse.\cite{wiki:MichelFoucault-de} Die Diskursanalyse untersucht im
allgemeinen den Zusammenhang von sprachlichem Handeln und sprachlicher Form im
Zusammenhang mit dem gesellschaftlichen Handeln, insbesondere institutionellen
Strukturen. Foucault stellte die traditionelle Geistesgeschichte in Frage, da
er sich nicht auf das erkennende Subjekt fokussierte, sondern sich auf
faktische Aussagen stützt, die die moderne Subjektivität erst hervorgebracht
haben. Sein Werk ist aber nicht prinzipiell eine neue Methode, sondern
theoretische Überlegungen für eine neue Art zu denken, auf eine
positivistische Art.\cite{wiki:Diskursanalyse-de}

Foucault war der Sohn eines Anatomieprofessors. Er brach aber bewusst mit der
Tradition der Familie ein Medizinstudium zu ergreifen und studierte
Philosophie und Psychologie in Paris. Nachdem er als Lehrer für Philosophie am
\emph{Collège de France} zugelassen wurde gehörten auch Paul Veyne und Jacques
Derrida zu seinen Schülern.  In seiner Forschung untersuchte er, wie Wissen
entsteht und Geltung erlangt, wie Macht ausgeübt wird und wie Subjekte
diszipliniert werden.\cite{wiki:MichelFoucault-de}

\section*{Text}
\label{sec-2}

Im Text beschreibt Foucault zunächst den historischen Hergang der Quarantäne
im Pestfall einer Stadt im 17. Jahrhundert. Damals gab es eine strikte
Hierarchie zur Überwachung der Quarantänemassnahmen vom Fürsten oder
Bürgermeister aus über sogenannte Intendanten und Syndices zu den einzelnen
Bewohnern herunter, welche die Ordnung allesamt unter Androhung des Todes
durchsetzten. Es gab keine Teilung der Gewalten und so war es die
absolutistische Willkür, die sich so über Verordnungen breit machen
konnte. Allerdings dies alles unter dem Vorwand des Schutzes der eigenen
Sicherheit.\cite{Foucault1977} Bezeichnenderweise ist dies auch heute noch das
gängige Argument um Überwachungsstrukturen zu legitimieren. Heute ist es
einfach nicht die Pest oder eine andere Seuche die den Agressor von aussen
darstellt, sondern häufig das Wort Terror oder Terroristen. Die Verschärfung
des Überwachungsgesetzes in der Schweiz im Herbst 2016 wurde hauptsächlich so
argumentiert.\cite{bundesrat}

Foucault beschreibt weiter, dass damals systematisch alle Einwohner erfasst
und Verzeichnet wurden ("`lückenloses Registrierungssystem"'
\cite{Foucault1977}), und alle täglich einen wahrheitsgemässen Bericht abgeben
mussten. Die Umstände gingen so weit, dass selbst die grundlegende
medizinische Versorgung auf dem Dienstweg der Hierarchie geschehen musste, um
nicht irgendwelche \emph{Geheimnisse} haben zu können. Die so etablierte Ordnung
schrieb allen ihren Platz (physisch), Körper und sogar sein Gut
vor.\cite{Foucault1977} In diesem Sinne ist das erfassen vieler persönlicher
Daten nicht erst ein Phänomen unserer Gesellschaft, es klärt sich aber noch
nicht, ob dies eine gute oder sinnvolle Sache ist.

Danach geht Foucault auf den Unterschied der Ordnungsstruktur bei der Pest und
dem einfachen Aussetzen bei Lepra ein, und bringt diesen mit politischen
Grundideen in Verbindung. Das eine ideal (Lepra) strebt nach einer \emph{reinen}
Gesellschaft, während das andere (Pest) eine \emph{disziplinierte} Gesellschaft
anstrebt. Er nennt es gar den "`Traum von einer disziplinierten Gesellschaft"'
\cite{Foucault1977}, wo es darum geht, die Beziehungen zwischen den einzelnen
Individuen der Gesellschaft zu kontrollieren und zu entflechten. Heute gienge
dieser Anspruch vermutlich viel zu weit, doch ist es mindestens bemerkenswert,
dass auch in den heutigen Regierungsformen, auch in der westlichen Welt,
ebenfalls ein Ideal der Angstherrschaft ist. Es wird vielleicht nicht mit der
Todesstrafe gedroht, doch ist mein generelles Erleben für das Unterlassen von
verschiedenen Handlungen daher geprägt, dass es immer darum geht, angedrohte
Konsequenzen zu vermeiden. Der Staat nimmt sich auch heute noch die Frechheit
heraus, über einzelne zu werten und zu urteilen, sie zu kategorisieren und
gegen den eigenen Willen zu irgendwelchen Dingen oder in irgendwelche
Institutionen zu zwingen.

Als letzten Teil beschreibt Foucault dann das \emph{Panopticon}, die
architektonische Manifestation des Überwachungsgedankens. Das Panopticon
funktioniert nur auf die Weise, dass es einen Überwachten und einen Überwacher
gibt, wobei der Überwachte gesehen wird, gleichzeitig aber nicht sehen kann ob
er gesehen wird. Es ist eine einseitige Überwachung. Gleichzeitig muss, damit
das Panopticon überhaupt funktioniert der Überwachte wissen, dass er jederzeit
überwacht werden kann. Er wird nicht mit physischer Gewalt zu einem Verhalten
gebracht, sondern durch den inneren Druck und die Ohnmacht in dieser Situation
aus der er sich nicht befreien kann. Dies geht so weit, dass man am Ende gar
keinen Überwacher mehr braucht, da es alleine schon reicht, dass der
Überwachte glaubt, dass die Möglichkeit besteht, dass er eventuell überwacht
werden könnte.

Durch diese einseitige Überwachungsstruktur wird das Individuum zum reinen
Objekt der Information aber niemals zum Subjekt einer
Konversation.\cite{Foucault1977} Auf der anderen Seite ist die Macht des
Überwachers auch nicht an dessen Subjekt gebunden. Die Macht entsteht einzig
durch die Institutionalisierung der Überwachung, wobei es egal ist, wer denn
gerade am Drücker ist.

Das Panopticon ist auf verschiedene Weise ein Versuchslaboratorium. Auf der
einen Seite beschreibt Foucault, dass man das physische Panopticon als
Versuchslaboratorium für Menschen benutzen kann, ich sehe aber auch den
Versuch, der sicherlich nicht beabsichtigt ist ablesen zu können, wie weit
eine willkürliche Gewalt gehen kann, bevor sich eine komplette Gesellschaft
dagegen auflehnt. Ich stelle mir zwei Fragen bezüglich dieses Textes:

\begin{itemize}
\item Sind wir heute in einer Art \emph{globalem Panopticon} ausgestellt?

Diese Vermutung liegt nahe, da jedem bekannt ist, dass wir permanent
einseitig überwacht werden, und wir immer noch keine Ahnung haben, oder auch
haben können, wie weit dies führt. Wir sind im Seminar permanent am
überlegen, was es für uns bedeutet, überwacht zu werden.

Ein weiterer Punkt hierzu ist für mich auch die Erkenntnis, dass im
Panopticon die effektive erzieherische Gewalt vom Individuum selber kommt,
das in der Angst, überwacht zu werden sich so verhält, als dürfe es nur im
Willen des Überwachers handeln. Dies erlebe ich in einer ähnlichen Form auch
heute bei uns. Wenn ich überwacht werde, hüte ich mich automatisch von mir
selber aus davor, auf irgendeine \emph{Blacklist} genommen zu werden, indem ich
mich mit diversen kritischen Inhalten nicht offensichtlich in Verbindung
bringen lasse.

\item Zum zweiten frage ich mich, was es eigentlich bedeutet Teil einer
Gesellschaft zu sein?

Im Panopticon sind alle Individuen physisch isoliert von den anderen. Dies
ist in unserer Gesellschaft sicherlich nicht der Fall. Aber inwieweit
ermöglicht diese Gegebenheit, dass wir nicht zu panoptischen Opfern werden?
Wenn ich mir meine Gedanken zum globalen Panopticon anschaue, scheint dies
mindestens für die Wirkungsstruktur des Überwachungsapparates keinen
wesentlichen Einfluss zu haben. Die Informationen die es zu ermitteln und
überwachen gilt sind vielleicht einfach komplexer, da sie nicht ohne
Einflüsse von anderer Seite existieren.

Was bedeutet es, Teil einer panoptischen Gesellschaft zu sein? Wenn die
gesamte Gesellschaft keine Geheimnisse mehr haben kann, kann sie sich
weiterentwickeln oder ist sie zur Stagnation im Status Quo verdammt?
\end{itemize}

\bibliographystyle{unsrt}
\bibliography{dasprivateinderdigitalengesellschaft}
% Emacs 25.3.1 (Org mode 8.2.10)
\end{document}
