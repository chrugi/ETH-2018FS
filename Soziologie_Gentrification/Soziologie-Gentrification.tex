% Created 2018-03-02 Fre 14:02
\documentclass[a4paper,11pt,ngerman]{article}
\usepackage[utf8]{inputenc}
\usepackage[T1]{fontenc}
\usepackage{fixltx2e}
\usepackage{graphicx}
\usepackage{longtable}
\usepackage{float}
\usepackage{wrapfig}
\usepackage{rotating}
\usepackage[normalem]{ulem}
\usepackage{amsmath}
\usepackage{textcomp}
\usepackage{marvosym}
\usepackage{wasysym}
\usepackage{amssymb}
\tolerance=1000
\usepackage{natbib}
\usepackage[linktocpage,pdfstartview=FitH,colorlinks,linkcolor=blue,
anchorcolor=blue,citecolor=blue,filecolor=blue,menucolor=blue,urlcolor=blue]{hyperref}
\usepackage{ngerman}
\author{Christian Sangvik}
\date{\today}
\title{Soziologie: Vor der Gentrification? Feldstudie der Rosengartenstrasse in Zürich}
\hypersetup{
  pdfkeywords={},
  pdfsubject={},
  pdfcreator={Emacs 25.3.1 (Org mode 8.2.10)}}
\begin{document}

\maketitle
\tableofcontents


\section{Administration}
\label{sec-1}

\begin{itemize}
\item Dozentur Soziologie
\item Keine Prüfung, aber \textbf{Anwesenheitskontrolle}
\item Gruppenarbeiten zu verschiedenen Abschnitten der Strasse.
\item Am Schluss wird eine Zeitung entwickelt.
\item Wahlfacharbeit möglich. Informationen auf der \href{http://www.soziologie.arch.ethz.ch/de/}{Dozenturwebseite}.
\end{itemize}

\subsection{Übersicht Programm}
\label{sec-1-1}

\begin{center}
\begin{tabular}{lrl}
23. Feb & 01 & Thematische Einführung\\
02. Mär & 02 & Gastinput und Gruppeneinteilung\\
09. Mär & 03 & Methoden und wissenschaftliches Arbeiten\\
16. Mär & 04 & Feedback zu Abgaben, Interviewleitfaden\\
23. Mär & -- & Seminarwoche\\
30. Mär & -- & Osterferien\\
06. Apr & -- & Osterferien\\
13. Apr & 05 & Feedback zu Abgaben, Präsentation Material\\
20. Apr & 06 & Tischkritik I\\
27. Apr & 07 & Tischkritik II\\
04. Mai & 08 & \textbf{Schlussabgabe}, Wahlfachauswertung, \textbf{Präsentation}\\
11. Mai & 09 & Kein Wahlfach, Zeitungsproduktion\\
18. Mai & 10 & Abschlussveranstaltung vor Ort\\
\end{tabular}
\end{center}

\subsection{Fragestellung}
\label{sec-1-2}

\begin{itemize}
\item Wie gestalten sich das alltägliche städtische Leben entlang der
Rosengartenstrasse und im Quartier heute?
\item Was zeichnet das Quartier aus? Welche urbane Qualitäten schätzen die
Bewohner und Gewerbebetreibenden?
\item Was würde eine Verkehrsberuhigung für sie bedeuten?
\end{itemize}

\subsection{Aufgabe}
\label{sec-1-3}

Wir untersuchen den anlaufenden Veränderungsprozess entlang der
Rosengartenstrasse und versuchen, andere Möglichkeiten einer Städtebaulichen
Entwicklung zu denken.

\subsection{Was erwartet Sie?}
\label{sec-1-4}

\begin{itemize}
\item Gruppenarbeit
\item Intensive Auseinandersetzung mit Gentrifizierung, einem aktuellen Thema von
Architektur und Städtebau
\item Starker Praxisbezug, konkretes Beispiel, gemeinsamer Output
\item Gastbeiträge
\item Erlernen und Anwendung der Werkzeuge für eine Feldforschung
(Interviewtechniken, Beobachtungen, Fotos, Text)
\item Zeit während des Semesters für Feldforschung, Stunden werden
kompensiert. Zeit selber einteilen!
\item Zeitung als gemeinsamer Output mit Ergebnissen des Wahlfachs, -
Intervention!
\end{itemize}

\subsection{Gruppeneinteilung}
\label{sec-1-5}

Gruppe 1

\section{Vorlesungen}
\label{sec-2}

\subsection{Thematische Einführung}
\label{sec-2-1}

\subsubsection{Rosengartenstrasse}
\label{sec-2-1-1}

Das Thema des Seminars ist die grössere Region der Rosengartenstrasse
zwischen Bucheggplatz und der Limmat in Wipkingen. Sie ist seit 1972 ein
Provisorium der Westtangente der Stadt Zürich. Ausserdem ist sie eine der
Wichtigsten Autostrassen in ganz Zürich. Die Verkehrslast steht bei über
50'000 Autos pro Tag, wbei der Grossteil auf den Individualverkehr
zurückzuführen ist.

Die Rosengartenstrasse schneidet Wipkingen in zwei Teile, die sehr
sporadisch miteinander verbunden sind. Bis in die 1970er Jahre war sie eine
normale Quartierstrasse. Die Struktur des Quartiers ist noch sehr dörflich,
aber mittlerweile durch die stark befahrene Strasse unterbrochen, und die
daraus resultierenden zwei Subquartiere sind stark getrennt
voneinander. Viele Anwohner sind der Strasse natürlich nicht sonderlich
positiv gegenübergestellt.

Auch die Lärmeindämmungsvorkehrungen tragen weiter zur Abtrennung der
einzelnen Teilquartiere bei.

Die neue Verkehrsplanung sieht vor, diesen Umstand zu ändern. Geplant ist
ein Tunnel und eine neue Tramlinie durch das Quartier. Der Tunnel kann aber
frühestens 2030 gebaut werden und das Tram wird nochmals zwei Jahre länger
dauern. Die dafür gerechneten Kosten sind immens (aktuell bei rund einer
Milliarde Franken) und steigen ständig.

Der Tunnel führt in der Vision direkt vom Milchbucktunnel zur Limmat. Der
Autotunnel wird eine Schlaufe machen müssen um die Steigung für die
Fahrzeuge einzuhalten.

Das Projekt ist aber erst in der Projektphase und eine Realisierung steht
noch schwer zur Debatte.

Die Vorzüge sind aber klar abzusehen. Durch das Quartier würde dann eine
Langsamverkehrszone führen (Tempo 30). Der fehlende Grossverkehr
ermöglichte natürlich auch neue Städtebauliche Möglichkeiten.

Alleine die Ankündigung dieses Projektes ruft aber jetzt bereits erste
Investoren auf den Plan an der Rosengartenstrasse aufwertungen zu
betreiben. Zu erwarten sind stark steigende Mieten und eine
Gentrifizierung.

Die Grundeigentümerschaft entlang der Strasse ist sehr heterogen. Sie reicht
von Genossenschaften über private Grundbesitzer bis hin zu Stadtgrund. Es
gibt keine grossflächigen Besitztümer sondern quasi durchgehend kleine
Parzellen.

Welche Rolle den Genossenschaften zukommen würde ist fraglich, aber sie wären
möglicherweise der Schlüssel zu günstigeren Mieten, auch wenn sie im
Vergleich zum Status Quo natürlich ansteigen würden.

Aktuell entsteht von einer Stiftung ein Studentenwohnheim. Man sieht also
bereits eine Transformation im der Struktur, obwohl noch nichts konkretes
politisch auf dem Tisch steht.

\subsubsection{Weststrasse Zürich}
\label{sec-2-1-2}

Die Weststrasse teilt ein ähnliches Schicksal, was mit der Westumfahrung
begonnen hat. Pro Stunde waren damals circa 1000 Autos und 100 Lastwagen
gezählt.

Heute prägt das Quartier einen völlig anderen Charakter. Öffentliche Räume
sind die Priorität und das Gebiet ist sehr fussgängerfreundlich ausgelegt.

Viel Gewerbe, das aber auf den Verkehr ausgerichtet war sieht die
Entwicklung nicht nur positiv.

Vor der Strassenberuhigung wohnten ca. 1200 Menschen an der Strasse,
mehrheitlich ausländischer Herkunft.

\begin{itemize}
\item \href{https://www.srf.ch/play/tv/tagesschau/video/weststrasse-endlich-wieder-ohne-verkehr?id\%3D77866ac6-343f-4af1-b856-b4dbf1a56092}{Schweiz Aktuell, Bericht vom 2.8.2010}
\item \href{https://www.srf.ch/play/tv/schweiz-aktuell/video/weststrasse?id\%3D7a960266-9558-460f-882f-db7c572aa28e}{Schweiz Aktuell, Bericht vom 9.2.2011}
\end{itemize}

Die Wohnungen die in den umgebauten Bürogebäuden und Wohnhäusern entstanden
sind natürlich unerschwinglich für die damaligen Bewohner. Die Menschen
wurden regelrecht aus dem Quartier vertrieben. In jedem zweiten Haus wurden
den Bewohnern gekündigt.

Bewegungen, die Stadt dazu zu bewegen, dieser Entwicklung entgegenzutreten
blieben erfolglos. Die Stadt habe den privaten Eigentümern nicht
hineinzureden. Es existiere keine rechtliche Grundlage für dergleichen. Dies
sitmmt nicht ganz, aber die Stadt hat es sich so einfach gemacht uns sich
aus der Entwicklung herausgehalten.

Innert weniger Monaten wurde die Strasse komplett umgebaut. Die
demographische Verteilung hat sich stark verändert.

Im Gegensatz zur Rosengartenstrasse waren an der Weststrasse aber keine
Genossenschaften oder Grundstück der Stadt an der Strasse, was die Willkür
der privaten Investoren natürlich noch bestärkte.

Es gab und gibt Proteste von Seiten der Kommunisten und Sozialisten, die
sich gegen die Gentrifizierung gewehrt hat. Bis heute ist die Diskussion
noch nicht abgeflacht, und Gentrifizierung wird oft mit diesem Beispiel
diskutiert.

Die neue Weststrasse wird aber nicht das Zukunftsszenario der
Rosengartenstrasse werden.

\subsubsection{Die \emph{neue} Weststasse}
\label{sec-2-1-3}

Zeitungsartikel des Wahlfaches. \href{http://www.soziologie.arch.ethz.ch/_DATA/90/FINAL_DieNeueWeststrasse_160517.pdf}{Online Verfügbar}.

Es wird am Schluss dieses Wahlfaches eine ähnliche Publikation geben.

\subsection{Inputvortrag zur Verkehrsplanungsgeschichte in Zürich, Daniel Weiss, gta}
\label{sec-2-2}

\subsubsection{Wipkingen und die Rosengartenstrasse, Ein Stück Stadtentwicklungsgeschichte}
\label{sec-2-2-1}

Wipkingen war bis mitte des 19. Jh. bäuerlich und ärmlich geprägt. Ab 1880
wurde Zürich immer wichtiger aufgrund der Entwilckung des
Eisenbahnknotenpunkts.

1872 wurde in Wipkingen eine erste Brücke über die Limmat gebaut, und somit
mit der Zürcher Industrie verbunden. Es kam zu einer ersten
verstädterung. Gebaut wurden aber eher ärmliche Mietskasernen. Die
Einwohnerzahl stieg rapide, aber nicht die Steuereinnahmen. So konnte
Wipkingen die Infrastruktur nicht halten und verschuldete sich
zusehends.

Mit dem Zusammenschluss mit der Stadt änderte sich dies. Die Bevölkerung
explodierte. 1892 wurde die Nordstrasse in Betrieb genommen und ein neues
Schulhaus gebaut. Die Stadt Zürich übernahm nach der Eingemeindung die
Schulden. 1898 Wurde Die Strassenbahn nach Wipkingen verlängert und bis zur
Nordbrücke geführt. Der untere Dorfkern musste neuen Strassen weichen. Diese
Entwicklung fand über einen sehr kurzen Zeitraum von rund 15 Jahren statt.

1934 wurde dann auch der obere Dorfkern geschleift zugunsten der
Rosengartenstrasse. Wipkingen erhält eine neue Zentralität am neu erstellten
Bahnhof.

Die Bautätigkeiten im Zentrum von Wipkingen war vor allem durch private
Investoren geprägt. Die Stadt baute vor Allem Strassen. Wipkingen wurde
immer mehr richtung Ober- und Unterstrass orientiert und entwickelte sich
vor allem in diese Richtung.

Um die Stadtentwicklung zu kontrollieren Baute die Stadt weitere
Infrastruktur und ermöglichte es diversen Genossenschaften unter Auflagen zu
bauen.

Die Bedeutung der Rosengartenstrasse kommt daher, dass sehr lange die Brücke
zwischen Escher-Wyss platz un Wipkingen die einzige befahrbare Strasse war,
die die Industrie mit Wipkingen verband.

Zürich plante eine Millionenstadt zu werden und plante
Satellitenstrassen. Mit dem Beschluss von 1958 zu einem flächendeckenden
Netz an Nationalsatrassen wurde aber der Druck auf die Bebauung schwächer,
da die Menschen von weiter her persönlich mobil anreisen konnten.

Es entstanden grosse Verkehrsplanerische Massnahmen, wovon aber viele nicht
umgesetzt wurden.

Geplant wurde aber bereits der Cityring. Es hätte einen grossen
Umfahrungsring und Durchfahrtstangenten geben sollen. Das Kernstück der
Verkehrsplanung war aber das grosse "Express-Y". Von Norden her durch den
neuen Milchbucktunnel direkt zum Letten und im Westen der Limmat entlang
ebenfalls zum Autobahnkreuz Letten. Nach süden hin sollte das Y bis zur
Üetlibergtangente und den Seetunnel münden.

Der Verkehr wuchs schneller als die Infrastruktur, deshalb mussten
Überbrückungslösungen hinhalten. Das Provisorium wurde 1982 in Betrieb
genommen. Die Stimmung in der Bevölkerung und bei Fachleuten hatte sich nun
aber verändert.

Zürich schrumpfte und die Leute zogen in die Agglomerationen. Es wurde eine
Abkehr von den Planungsprämissen gefordert. Die Verbetonierung der
Landschaft und Zersiedelung wurde angeprangert.

Es wurde eine Verkehrsberuhigung vorgeschlagen. Das ZAS (Zürcher
Arbeitsgruppe für Städtebau), eine Gruppe von diversen Akteuren, unter
Anderen auch Rolf Keller, wollte die Blechlawinen aus der Stadt
verbannen. Bereits 1960 wurde die Verkehrsplanung und deren Auswirkungen,
vor Allem aber das grosse Y angegriffen. Es wurde auch politisch druck
gemacht, und in der Bevölkerung wurde mittels einer Initiative das Y
verhindert.

Die ganzen Diskussionen zogen sich sehr in die Länge. Die Bewegung "Käi
Autobahn dur d' Stadt" kam auf. Der Bau des Milchbucktunnels hatte aber
schon lange begonnen. So kann man das Y auch heute noch in der Stadt lesen,
obwohl keine Autobahn hierdurch führt.

Politisch war die Stadt und der Kanton nicht immer der gelichen Meinung und
die Beschlüsse der Stadt wurden mehrmals vom Kanton überstimmt. Die Hürden
des Föderalismus lässt sich in dieser Planungsgeschichte gut ablesen.

Als letztes Teilstück der Westtangente rückt nun heute die
Rosengartenstrasse in den Fokus. Was könnte eine Verkehrsberuhigung heute
bewirken?

\section{Arbeitsorganisation}
\label{sec-3}

\subsection{{\bfseries\sffamily TODO} Zeitung lesen auf 2. März}
\label{sec-3-1}
% Emacs 25.3.1 (Org mode 8.2.10)
\end{document}
