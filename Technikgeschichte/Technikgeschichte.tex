% Created 2018-05-08 Di 11:35
\documentclass[a4paper,ngerman,11pt]{scrartcl}
\usepackage[utf8]{inputenc}
\usepackage[T1]{fontenc}
\usepackage{fixltx2e}
\usepackage{graphicx}
\usepackage{longtable}
\usepackage{float}
\usepackage{wrapfig}
\usepackage{rotating}
\usepackage[normalem]{ulem}
\usepackage{amsmath}
\usepackage{textcomp}
\usepackage{marvosym}
\usepackage{wasysym}
\usepackage{amssymb}
\tolerance=1000
\usepackage{natbib}
\usepackage[linktocpage,pdfstartview=FitH,colorlinks,linkcolor=black,
anchorcolor=black,citecolor=black,filecolor=black,menucolor=black,urlcolor=black]{hyperref}
\usepackage{ngerman}
\usepackage{url}
\addtokomafont{disposition}{\rmfamily}
\author{Christian Sangvik}
\date{\today}
\title{Technikgeschichte für Neugierige}
\hypersetup{
  pdfkeywords={},
  pdfsubject={},
  pdfcreator={Emacs 25.3.1 (Org mode 8.2.10)}}
\begin{document}

\maketitle
\tableofcontents


\section{Administration}
\label{sec-1}

David Gugerli

\begin{itemize}
\item Lektürelastig

Kurs auf \href{https://moodle-app2.let.ethz.ch/course/view.php?id\%3D4214}{Moodle}.
\end{itemize}

\subsection{Fragen}
\label{sec-1-1}

\begin{itemize}
\item Technik als Wunschmaschine
\item Technik als Katastrophe
\item Technik als Innovation
\item Technik als Assoziation
\end{itemize}

\subsection{Vorbereitung}
\label{sec-1-2}

Pro Sitzung den Haupttext lesen und einmal im Semester noch den
Sekundärtext.

Kommentar liefern. Abgabe jeweils Sonntags.

\subsection{Übersicht Programm}
\label{sec-1-3}

\begin{center}
\begin{tabular}{lrl}
27. Feb & 01 & Einführung\\
06. Mär & 02 & Wunschmaschine I\\
13. Mär & 03 & Wunschmaschine II\\
20. Mär & 04 & Wunschmaschine III\\
27. Mär & 05 & Wunschmaschine IV\\
 & -- & \\
10. Apr & 06 & Katastrophe I\\
17. Apr & 07 & katastrophe II - \textbf{Primärtext}\\
24. Apr & 08 & Innovation I\\
 & -- & \\
08. Mai & 09 & Innovation II\\
15. Mai & 10 & Assoziation I - \textbf{Sekundärtext}\\
22. Mai & 11 & Assoziation II\\
29. Mai & 12 & Bilanz\\
\end{tabular}
\end{center}

\section{Vorlesungen}
\label{sec-2}

\subsection{Einführung}
\label{sec-2-1}

\subsection{Wunschmaschine}
\label{sec-2-2}

Was ist die Wunschmaschine? Die Wunschmaschine ist ein Konstrukt, das wir uns
selber gegeben haben. Wir haben ein Raster mit ungefähr sechs Punkten
entwickelt.

Wir müssen die Wunschmaschine nicht zwingend mit der Physischen Welt zu
tun. Es geht nicht zwingend um eine materielle Maschine im eigentlichen Sinn.

Wir vergegenständlichen die Dinge mit der Sprache, die wir dafür verwenden.

\subsection{Wunschmaschine I}
\label{sec-2-3}

\subsection{Wunschmaschine II}
\label{sec-2-4}

\subsubsection{Fragen}
\label{sec-2-4-1}

\begin{itemize}
\item Normierung, Standardisierung
\item Selektion, Filter
\item Politics of Artefacts
\end{itemize}

\subsubsection{Normierung und Standardisierung}
\label{sec-2-4-2}

Rüstungsgüter dezentral herstellen. Dezentral hergestellte Güter muss man
aber auch kombinieren können. Dies fordert eine Normierung. Die Idee ist
alt, aber wurde durch die Kriegsproduktion gepusht.

Der Bau selber (Bauweise und Abfolge) sind normiert. Erst die
organisatorische und materielle Komponente bringen die Norm. Ausserdem
gehört die Kundschaft dazu.

Lebenskultur und Mobilitätskultur. Auto produkt einer hochgradigen
Normierung. Gesetze die den Tagesablauf und die möglichkeiten der sozialen
Praxis werden sind auch Normen.

Die norm flexibilisiert, eröffnet tauschmöglichkeiten, schränkt aber
gleichzeitig enorm ein. Dies ist auch ein Punkt der Politik der Norm.
(Austauschbar, ja aber nur, wenn es vom gleichen Typ ist)

Die Norm ist eng gekoppelt an Industrialisierungsprozess. Der Wohnungsbau
wird industrialisierbar.

Finanzierung als Norm

Verträge sind daher kompliziert

Normraum für das Wohnen ist ausserhalb der Stadt.

\subsubsection{Selektion}
\label{sec-2-4-3}

Normierende Wirkung. Die Norm selber ist bereits eine stake Selektion.

Rassensegregation. Rassen aber auch Altersgruppen getrennt. Junge Familien
mit GIs. Eigenes Haus mit Garten.

Selektionen werden nur weitergeführt und verstäkrt diese (Alle Formen sind
bereits vorher in der amerikanishen Gesellschaft). Es soll eine Klassenlose
Gesellschaft nach marxistischem Vorbild geschaffen werden. Alle sind gleich
innerhalb des Clusters.

\subsubsection{Politik}
\label{sec-2-4-4}

\subsection{Wunschmaschine III}
\label{sec-2-5}

\subsection{Wunschmaschine IV}
\label{sec-2-6}

\subsection{Katastrophe I}
\label{sec-2-7}

\subsubsection{Technik als Katastrophe}
\label{sec-2-7-1}

Das Gegenteil technischer Wunschmaschinen

\begin{itemize}
\item Projektion

\begin{itemize}
\item Angst
\item Unsicherheit, Ungewissheit
\end{itemize}

\item Mis en place

\begin{itemize}
\item Durcheinander
\item Untersuchung Regeln
\item Risikobereitschaft
\item In Sicherheit fühlen
\end{itemize}

\item Bricolage

\begin{itemize}
\item Rechtfertigung
\item Vorbereiten auf Ernstfall, potentielle Massnahmen
\end{itemize}

\item Kohärenz, Statik

\begin{itemize}
\item Titaker Zusammenbruch
\item Helden
\item "starke Männer"
\item Kausalität (Glaubwürdige Argumentation)
\end{itemize}

\item Fokussierung

\begin{itemize}
\item Opfer
\item auf Folgen
\item Unschuld/ Schuld
\end{itemize}

\item Narrativ

\begin{itemize}
\item Verengung (statt offener Erwartungshorizont)
\end{itemize}

\item Norm

\begin{itemize}
\item Katastrophenskala
\item Soziale Norm bei Unfällen
\item Tragik
\end{itemize}

\item Selektion

\item Politics
\end{itemize}

\begin{center}
\begin{tabular}{l}
Kontrollverlust\\
Lernprozess\\
Verantwortungszuschreibung\\
\end{tabular}
\end{center}

\begin{center}
\begin{tabular}{lll}
Wir & Katastrophe & Technik\\
\hline
Mensch & Unfall / Zwischenfall & Maschinen\\
Konsumenten & Gefahr & Verkehr\\
Experten & Risiko & Dienstleistung\\
Erfinder & Normalität & Entwicklung / Patent\\
Opfer & Kontrolle & Produkt\\
Ingenieure &  & Arbeitsmethode\\
Arbeiter &  & Prozess / Verfahren\\
Unternehmer &  & Apparatur\\
Operateure / Maschinisten &  & System\\
User &  & \\
Bürger &  & \\
Anbieter &  & \\
Bewohner / Bevölkerung &  & \\
Beamte &  & \\
Politiker &  & \\
Probanden &  & \\
\end{tabular}
\end{center}


\subsection{Katastrophe II}
\label{sec-2-8}

Im Gegensatz zu den Wunschmaschinen schreiben wir uns in Rage, wenn wir über
die Katastrophe schreiben.

Auch die Abgrenzung von Unfall zu Katastrophe macht uns noch Mühe.

Jede Katastrophe hat Folgen, ist spektakulär. Bei Giedion ist dies nicht der
Fall. Wir wissen um die Katastrophe und machen da auch mit. Was ist das
katastophale an dieser Killerfabrik?

Ist die Mechanisierung eine Art Erdbeben, gegen das wir nichts tun können?
Nein. Woher kommt dann die Mechanisierung? Ist sie bloss eine Anpassung an
neue Umstände?

Rollen in diesem Disaster

\begin{itemize}
\item Ingenieure

Reichen die Patente ein

\item Ökonomen

\item Fleischesser

\item Operateure

\item Staaten

\item Organisationen

\item Arbeiter
\end{itemize}


Entweder sagt man, die Mechanisierung übernimmt das Kommando und das endet
dan in Auschwiz, oder man sagt, die Maschine ist kontrollierbar, und der
Mensch

Zeigt uns die Katastrophe, was das genuin menschliche ist?

Die Maschine ist nicht aus technischer Sicht ausser Kontrolle, sondern die
Katastrophe ist ethischer Natur.

Der Mensch wird in die Maschine eingebaut. Das Verbrauchsmaterial Mensch wird
optimal in den Produktionsprozess eingegliedert.


\subsubsection{Roboter}
\label{sec-2-8-1}

Sind die Roboter vergleichbar, mit einer avancierten mechanischen Maschine,
der überdies geanu gleich funktioniert? Ist die Katastrophe der
Roboterisierung dieselbe wie bei der Mechanisierung?

Wir haben immernoch zu kämpfen mit dem abstrakten Menschenkonzept, gegen
welches der Roboter aufstehen muss.


Technik macht Massstabssprung. Aus klein wird gross.

\subsection{Innovation I}
\label{sec-2-9}

\subsection{Innovation II}
\label{sec-2-10}

\begin{center}
\begin{tabular}{llll}
\textbf{Invention} & \textbf{Innovation} & \textbf{Diffusion} & \textbf{Wissenschaftspolitik}\\
\emph{Technikgeschichte} & \emph{Wirtschaftsgeschichte} & \emph{Technikgeschichte}\\
Erfinder &  & Wo die grosse Wirkung\\
Maschinen &  & sich bemerkbar macht\\
Ingenieure &  & Massenmarkt\\
Tüftler / Pioniere &  & Ganze Welt\\
\end{tabular}
\end{center}

\subsection{Assoziation I}
\label{sec-2-11}

\subsection{Assoziation II}
\label{sec-2-12}

\subsection{Bilanz}
\label{sec-2-13}
% Emacs 25.3.1 (Org mode 8.2.10)
\end{document}
